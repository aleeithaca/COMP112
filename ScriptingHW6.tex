\documentclass{article}
\usepackage{fancyhdr}
\usepackage{float}
\usepackage[margin=1in]{geometry}
\usepackage{multicol}
\usepackage{url}
\usepackage{hyperref}
\usepackage{amsmath, amssymb, amsfonts}
\usepackage{graphicx}
\usepackage{xcolor}
\usepackage{subcaption}
\hypersetup{
    colorlinks=true,
    linkcolor=blue,
    urlcolor=blue,
}
\newcommand{\AName}{Scripting HW 6}
\newcommand{\ALength}{50 - 90 minutes}
\newcommand{\ADate}{11/07/2024}
\newcommand {\ADueDate}{11/12/2024}
\newcommand {\AAcceptDate}{11/14/2024}
\newcommand{\AValue}{100}
\pagestyle{fancy}
\fancyhead{}
\fancyhead[L]{\begin{tabular}{l}
	{\AName} \\
	{\ALength} \\
	\end {tabular}}
	
\fancyhead[C]{\begin{tabular}{|c|c|c|}
  \hline
  \textbf{Date Posted:} & \textbf{Date Due:} & \textbf{Accepted Until:} \\
  \hline
  \ADate & \ADueDate & \AAcceptDate \\
  \hline
  \end {tabular}}
  
\fancyhead[R]{\begin{tabular}{r}
	{COMP 110} \\
	{Fall 2024} \\
	\end {tabular}}

\begin{document}
\textbf{Welcome to \AName!  By the end of this lesson, Students Will be Able To...}
\begin{itemize}
    \item Modify existing functions to use lists instead of individual variables
\end{itemize}


\section*{Part 1: Get the file}
\begin{itemize}
    \item Open the start file:  \url{https://docs.google.com/spreadsheets/d/1CaVbd0uFHHNraiPmc7kOH9TCbseQVCgbHF8HTPZFDhI/edit?usp=sharing}
    \item Once opened, click File, then Make a Copy.  Place this copy into your COMP-190 folder.
    \item Once moved, open the file and Apps Script.  Within, you should see four functions - main, minOfFour, maxOfFour and magic.
\end{itemize}

\section*{Part 2: main function}
Let's review our variables and our list.  We will then move into modifying functions.
\begin{itemize}
	\item Go to the main function.  Here, you will find four variables - a, b, c and d.  Each one is assigned a number.
	\item There is a fifth variable called list, and the list contains four objects - a, b, c and d.
	\item Let's review each individual position in the list, and the associated variable.
	\begin{itemize}
		\item a - list[0]
		\item b - list[1]
		\item c - list[2]
		\item d - list[3]
	\end{itemize}
\end{itemize}

\section*{Part 3: minOfFourList function}
Let's modify your minOfFour function to use a list of four values, as opposed to using four variables
\begin{itemize}
    \item Copy the minOfFour function.  Paste it, and rename the function to minOfFourList.
    \item Change the function arguments.  Instead of a, b, c and d, use just one - list.
    \item Modify your if statements.  Replace every a, b c and d variable with the associated list position.  See part 2 for inspiration!
    \item You will also need to modify your return statement.
    \item Once you have this completed, test this function using your main function.  You should get the same results as your minOfFour function.
\end{itemize}

\section*{Part 4: maxOfFourList function}
Let's now create our maxOfFourList function.
\begin{itemize}
    \item Copy the minOfFourList function.  Paste it.  Rename it to maxOfFourList.  Modify the arguments to use only one - list.
    \item You need to modify one thing in this function - the return statement.
    \item Once you have this completed, test this function using your main function.  You should get the same results as your maxOfFour function.
\end{itemize}

\section*{Part 4: magicList function}
Let's modify our magic function from SWAT 5 to use lists.
\begin{itemize}
    \item Copy the magic function.  Paste it.  Rename it to magicList.  Modify the arguments to use only one - list.
    \item Make the necessary changes to make this function work with a list, rather than 4 variables.
    \item Test it, and make sure it works.
\end{itemize}

\section*{Submission}
When submitting the assignment, ensure that the settings are changed so that anyone with the link can view.
\begin{figure}[H]
  \centering
  \includegraphics{submission.png}
  \caption{Submission Settings}
\end{figure}

\section*{How this is graded}
This assignment is worth \AValue \ points. You will achieve all \AValue \   points if the following things are completed:
\begin{itemize}
    \item A minOfFourList function (30 pts)
    \item A maxOfFour function (30 pts)
    \item A magicList function (40 pts)
\end{itemize}
\end{document}