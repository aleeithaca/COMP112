\documentclass{article}
\usepackage{fancyhdr}
\usepackage{float}
\usepackage[margin=1in]{geometry}
\usepackage{multicol}
\usepackage{url}
\usepackage{hyperref}
\usepackage{amsmath, amssymb, amsfonts}
\usepackage{graphicx}
\usepackage{xcolor}
\usepackage{subcaption}
\hypersetup{
    colorlinks=true,
    linkcolor=blue,
    urlcolor=blue,
}
\newcommand{\AName}{Scripting SWAT 8}
\newcommand{\ALength}{50 - 90 minutes}
\newcommand{\ADate}{11/14/2025}
\newcommand {\ADueDate}{11/19/2025}
\newcommand {\AAcceptDate}{11/21/2025}
\newcommand{\AValue}{100}
\pagestyle{fancy}
\fancyhead{}
\fancyhead[L]{\begin{tabular}{l}
	{\AName} \\
	{\ALength} \\
	\end {tabular}}
	
\fancyhead[C]{\begin{tabular}{|c|c|c|}
  \hline
  \textbf{Date Posted:} & \textbf{Date Due:} & \textbf{Accepted Until:} \\
  \hline
  \ADate & \ADueDate & \AAcceptDate \\
  \hline
  \end {tabular}}
  
\fancyhead[R]{\begin{tabular}{r}
	{COMP 110} \\
	{Fall 2024} \\
	\end {tabular}}

\begin{document}
\textbf{Welcome to \AName!  By the end of this lesson, Students Will be Able To...}
\begin{itemize}
    \item Use the push method to add objects to a list.
    \item Write a function to see if a word is a palindrome.
\end{itemize}


\section*{Part 1: Create the file}
\begin{itemize}
    \item In your COMP-190 folder, create a new Google Sheets file. Rename this file to SWAT 8.  Open Apps Script.  Rename the project to SWAT 8.
\end{itemize}

\section*{Part 2: List Methods}
We know about list properties, which are characteristics of the list.  Let us now learn about methods.
\begin{itemize}
    \item Methods are actions you can do TO a list.  Methods have a similar syntax to properties - usually listname.method()
    \item For our purposes, we will be using the .push method - this is how we add objects into our list.  These are added to the end of the list.
    \item Create a main function, with no arguments.
    \item Within main, create a variable that is a blank list.  var list = []
    \item On the next line, type the following:  list.push("b")
    \item On the next line, use console.log to show the list variable.  Run your main function - the output should be this:  ["b"]
    \item Add a new line after list.push("b").  Push the value "a" into your list.
    \item Run the main function again.  Your output should be this:  ["b", "a"]
    \item Push "n" into your list.  Run the main function again.  Your output should be this:  ["b","a","n"]
    \item Push the remaining "a", "n", and "a" into the list.  Your output at the end should be ["b","a","n","a","n","a"]
\end{itemize}

\section*{Part 7: reverseList function}
Let's take a list, and reverse it!
\begin{itemize}
    \item The purpose of this function is to take a list, create a new list with the sent lists objects in reverse order, and send back to reserved list.
    \item For example, if your list is ["b","a","n","a","n","a"], the reserved list would be ["a","n","a","n","a","b"].
    \item Create a new function called reserveList.  Send this one argument - a list.
	\item We will need a new list, so we can push our individual objects into it.  To create a new blank list, simply use [ ]
	\item var newlist = [ ]
	\item We will need a loop variable, that starts at the end of our sent list.  var i = list.length -1
	\item We will need a loop that starts at the end of our sent list and ends at the beginning (aka position zero).
	\item while(i $>=$ 0)
	\item Within the loop, use newlist.push(list[i]) to take each object from the sent list and push it into the newlist.
	\item Don't forget to decrement your loop variable!
	\item Outside the loop, return the newlist.
	\item Test this using your main function.
\end{itemize}

\section*{Part 8: Palindrome}
A palindrome is a word that is the same backwards as it is forwards.  
\begin{itemize}
	\item An example palindrome is otto, level, and racecar - the words are spelled the same forwards and backwards.
	\item This can be done in Sheets, example here:  \url{https://chandoo.org/wp/palindrome-check-excel-formula/}
	\item Like before, it's not intuitive, easy to understand, and the syntax is awful.  Let's write our own function instead!
\end{itemize}

\section*{Part 9: isPal function}
The isPal function should return true if a word is a palindrome, and false if it is not.
\begin{itemize}
	\item Our isPal function will need to take a list, get the reverse of that list, then run through and compare each position of the lists together.
	\item Create a new function called isPal.  Send this function one argument - a list.
	\item Create a new variable called newlist.  This variable should be assigned to the reverse of the sent list.  Use the reverseList function to do so.
	\item var newlist = reverseList(list)
	\item At this point, we have two lists - list is our original list, and newlist is that list, reversed.
	\item If this is truly a palindrome, each position in list and newlist should be the same.
	\item It is only when they are not the same (not equals) that we don't have a palindrome.
	\item What we need to do is loop through each position and compare the list positions together.  As each list is the same length, we need only one loop.
	\item You'll need a loop variable that starts at zero.
	\item The loop should start at zero and run the length of the list.
	\item Within the loop, use if to see if the position of the list and newlist are not equals.  If this is true, return false.
	\item Outside the loop, return true.  Once we have run through the loop, and false was not returned, that means everything in each list was the same.
\end{itemize}

\section*{Submission}
When submitting the assignment, ensure that the settings are changed so that anyone with the link can view.
\begin{figure}[H]
  \centering
  \includegraphics{submission.png}
  \caption{Submission Settings}
\end{figure}

\section*{How this is graded}
This assignment is worth \AValue \ points. You will achieve all \AValue \   points if the following things are completed:
\begin{itemize}
    \item A reverseList function (50 pts)
    \item An isPal function (50 pts)
\end{itemize}
\end{document}