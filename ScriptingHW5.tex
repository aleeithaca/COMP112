\documentclass{article}
\usepackage{fancyhdr}
\usepackage{float}
\usepackage[margin=1in]{geometry}
\usepackage{multicol}
\usepackage{url}
\usepackage{hyperref}
\usepackage{amsmath, amssymb, amsfonts}
\usepackage{graphicx}
\usepackage{xcolor}
\usepackage{subcaption}
\hypersetup{
    colorlinks=true,
    linkcolor=blue,
    urlcolor=blue,
}
\newcommand{\AName}{Scripting HW 5}
\newcommand{\ALength}{50 - 90 minutes}
\newcommand{\ADate}{04/02/2025}
\newcommand {\ADueDate}{04/07/2025}
\newcommand {\AAcceptDate}{04/09/2025}
\newcommand{\AValue}{100}
\pagestyle{fancy}
\fancyhead{}
\fancyhead[L]{\begin{tabular}{l}
	{\AName} \\
	{\ALength} \\
	\end {tabular}}
	
\fancyhead[C]{\begin{tabular}{|c|c|c|}
  \hline
  \textbf{Date Posted:} & \textbf{Date Due:} & \textbf{Accepted Until:} \\
  \hline
  \ADate & \ADueDate & \AAcceptDate \\
  \hline
  \end {tabular}}
  
\fancyhead[R]{\begin{tabular}{r}
	{COMP 110} \\
	{Spring 2025} \\
	\end {tabular}}

\begin{document}
\textbf{Welcome to \AName!  By the end of this lesson, Students Will be Able To...}
\begin{itemize}
    \item Use a loop to "prove" the collatz conjecture
\end{itemize}


\section*{Part 1: Create the file}
\begin{itemize}
    \item In your COMP-190 folder, create a few Google Sheets file.  Rename this file to HW 5.  Open Apps Script.  Rename the project to HW 5.
\end{itemize}

\section*{Part 2: Collatz Conjecture Overview}
Similar to Kaprekar's Magic, the collatz conjecture is a mathematical idea that can be "proven" with loops.  More information here:  \url{https://en.wikipedia.org/wiki/Collatz_conjecture}
\begin{itemize}
    \item The premise of the collatz conjecture is simple:  Take any integer (no decimals!) greater than 1.  Find out if it is even.  If it is, multiply it by 2.  If it is not even, multiply the number by 3 and add 1.
    \item Take the resulting integer, and see if it is even.  If it is, divide by 2.  If it is not even, multiply it by 3 and add 1.
    \item Do this enough times, and eventually, the result will be 1 - NO MATTER WHAT NUMBER YOU START WITH.
\end{itemize}

\section*{Part 3: collatz function}
Let's write a function that will find out how many times a loop runs until the result is 1.
\begin{itemize}
    \item In Apps Script, write a new function called collatz.  Send this function a number, $n$
    \item You will need a loop but not a loop variable.  As we are finding out how many times the loop runs, set up a variable called count, initially set to 0.
    \item When setting up your loop, we want the loop to run while $n$ is not equal to 1.
    \item Within the loop, do the following:
    \begin{itemize}
    		\item Check to see if $n$ is even.  If it is, reassign $n$ to be $n$ / 2.
    		\item If $n$ is not even, reassign $n$ to be $n$ * 3 + 1.
    		\item Increment your count by 1.
    	\end{itemize}
    	\item Outside of your loop, return count.
    	\item Test it!  You should get results similar to the following:
    	\begin{figure}[H]
  		\centering
  		\includegraphics{swat_5_collatz}
  		\caption{collatz interations}
	\end{figure}
\end{itemize}

\section*{Submission}
When submitting the assignment, ensure that the settings are changed so that anyone with the link can view.
\begin{figure}[H]
  \centering
  \includegraphics{submission.png}
  \caption{Submission Settings}
\end{figure}

\section*{How this is graded}
This assignment is worth \AValue \ points. You will achieve all \AValue \   points if the following things are completed:
\begin{itemize}
    \item a collatz function (100 pts)
\end{itemize}
\end{document}