\documentclass{article}
\usepackage{fancyhdr}
\usepackage{float}
\usepackage[margin=1in]{geometry}
\usepackage{multicol}
\usepackage{url}
\usepackage{hyperref}
\usepackage{amsmath, amssymb, amsfonts}
\usepackage{graphicx}
\usepackage{xcolor}
\usepackage{subcaption}
\hypersetup{
    colorlinks=true,
    linkcolor=blue,
    urlcolor=blue,
}
\newcommand{\AName}{Scripting SWAT 13}
\newcommand{\ALength}{50 - 90 minutes}
\newcommand{\ADate}{12/10/2024}
\newcommand {\ADueDate}{12/12/2024}
\newcommand {\AAcceptDate}{12/17/2024}
\newcommand{\AValue}{100}
\pagestyle{fancy}
\fancyhead{}
\fancyhead[L]{\begin{tabular}{l}
	{\AName} \\
	{\ALength} \\
	\end {tabular}}
	
\fancyhead[C]{\begin{tabular}{|c|c|c|}
  \hline
  \textbf{Date Posted:} & \textbf{Date Due:} & \textbf{Accepted Until:} \\
  \hline
  \ADate & \ADueDate & \AAcceptDate \\
  \hline
  \end {tabular}}
  
\fancyhead[R]{\begin{tabular}{r}
	{COMP 110} \\
	{Fall 2024} \\
	\end {tabular}}

\begin{document}
\textbf{Welcome to \AName!  By the end of this lesson, Students Will be Able To...}
\begin{itemize}
    \item Turn rows into columns, columns into rows, transposition of ranges
\end{itemize}

\section*{Part 1: Create the file}
\begin{itemize}
	\item In your COMP-190 folder, create a new Google Sheets file.  Rename this file to SWAT 13.  Open Apps Script.  Rename the project to SWAT 13.
\end{itemize}

\section*{Part 2: TRANSPOSE function}
We know how to traverse a range with loops, doing things like displaying, or adding, or pushing those objects.  Now we will turn rows into columns, columns into rows, and transpose our data.
\begin{itemize}
	\item Let's look at the transpose function, what it does, and then apply those same concepts to Apps Script.
    \item In your sheet, add 15 or so random letters to the range A1:C5, creating a range that has 5 rows and 3 columns.
    \item In your sheet, somewhere else, write the following function:  =TRANSPOSE(A1:C1).
    \item When you press Enter, what happens?  It takes the row of characters you chose (A1:C1) and turned it into a column.
    \item In your sheet, somewhere else, write the following function:  =TRANSPOSE(A1:A5).
    \item When you press Enter, what happens?  It took that column of information (A1:A5) and turned it into a row.
    \item In your sheet, somewhere else, write the following function:  =TRANSPOSE(A1:C5).
    \item When you press Enter, what happens?  The rows are now columns, and the columns are now rows.
    \item Let's tackle each of these things individually, and write three functions that do these transpositions.
\end{itemize}

\section*{Part 3: main function}
Let's create our main function in Apps Script, and set up our range as it is on the sheet.
\begin{itemize}
    \item In Apps Script, create a new function called main.  Send this function no arguments.
    \item In main, set up your range.  Remember that this is a list, of lists.  Your range should look similar to the following example:
    \item var range = [ [ "a","b","c" ], [ "d","e","f ] ]  Note that this is only two rows, your example should have five.
    \item Once set up, use console.log to display the range.  Run main, and make sure it works!
\end{itemize}

\section*{Part 3: rowToColumn function}
Let's first tackle the first problem, turning a single row into a single column.
\begin{itemize}
    \item In Apps Script, create a new function called rowToColumn.  Send this function one argument - range.
    \item Let's review what a single row in a range looks like, as we will be using these functions from the spreadsheet.
    \item A single row is a list of lists, with only one list within.  It looks like this:  [ [ "a","b","c" ] ]
    \item A single column in a range looks like this:  [ ["a"], ["b"], ["c"] ]
    \item Here's what we need to do:  go through the row, and push each object in the row, as a row, into a list.
    \item For example, "a" becomes ["a"], "b" becomes ["b"], and "c" becomes ["c"].  How do we do this?
    \item First, we need a new blank list.  var newlist = [ ]
    \item If you said loop, you are correct!  Set up a loop variable, initially set to zero.
    \item Create a while loop that starts at zero and runs through the length of our first row (aka the number of columns).  It should look like this:
    \begin{itemize}
    		\item while(i < range[0].length)
    	\end{itemize}
    	\item Within the loop, do the following:
    	\begin{itemize}
    		\item into your newlist, push a row.  The contents of the row should be range[0][i].  Note that if your loop variable is not i, be sure to use your loop variable.
    		\item It should look like this:  newlist.push([range[0][i]])
    		\item Increment your loop variable!
    	\end{itemize}
    	\item Outside of the loop, return your newlist.  
    	\item Call this function from main.  To send one row from your range, you will have to send the row, enclosed in brackets.
    	\item For example, to send the first row, use [range[0]] as your argument.  It would look something like this:
    	\item console.log(rowToColumn([range[0]])
    	\item Also, test this from the spreadsheet.  Your rowToColumn function, when given the range A1:C1, should return the same output as your first TRANSPOSE function from part 2.
\end{itemize}

\section*{Part 4: selectColumn function}
Getting a single row out of a range is fairly simple.  Getting a single column is a bit trickier.  Let's try!
\begin{itemize}
    \item Create a new function called selectColumn.  Send this function two arguments - range and colnum.
    \item colnum is a number, and is which column you wish to see.  Note that columns, just like rows, start at 0. So, if you wish to see the first column, you'll need position zero.
    \item Like rowToColumn, you'll need a new blank list and a loop variable, initially set to zero.
    \item As you may have guessed, you will need a loop.  This loop will need to start at zero, and run through the number of rows (recall that this is range.length)
    \item Within the loop, do the following:
    \begin{itemize}
    		\item Into your newlist, push a new row.  This row should contain the range[loop variable][colnum].
    		\item Something like this:  newlist.push([range[i][colnum]])
    		\item Increment your loop variable!
    	\end{itemize}
    	\item Outside of the loop, return the newlist.
    	\item Test this from both the main function and the sheet, to make sure it works.
\end{itemize}

\section*{Part 5: columnToRow function}
Now that we can select and create a single column, let's turn that column into a row.
\begin{itemize}
    \item Create a new function called columnToRow.  Send this function one argument - range.
    \item We will need to loop through the number of rows, and push the first position of each row into a new list.
    \item You will need a new blank list and a loop variable, initially set to zero.
    \item You will need a loop that starts at zero and runs through the number of rows.
    \item Within the loop, do the following:
    \begin{itemize}
    		\item Into your new list, push a row.  This row should contain the contents of range[loop variable][0].
    		\item Increment your loop variable.
    	\end{itemize}
    	\item Outside of the loop, return your new list.  Be sure to return your new list as a range (for example, return [newlist]
    	\item Test this function from both main and the spreadsheet.
\end{itemize}

\section*{Part 6: transposeRange function}
Now let's use our newly created functions to write the transpose function, in Apps Script!
\begin{itemize}
    \item Create a new function called transposeRange.  Send this function one argument - range.
    \item In this function, we will need to do the following:
    \begin{itemize}
    		\item Select each individual column
    		\item Turn that column into a row
    		\item Push that row into a new list
    		\item Return that new list.
    	\end{itemize}
    	\item You'll need the following things:
    	\begin{itemize}
    		\item a new blank list.
    		\item A loop variable
    		\item A loop that starts at zero and runs the number of columns.
    	\end{itemize}
    	\item Within the loop, do the following:
    	\begin{itemize}
    		\item Use selectColumn to select each individual column.
    		\item Use columnToRow to turn that column into a row
    		\item Push the row into the new list.  Note, when doing this, recall that the columnToRow function will return a range, so when pushing the row into the new list, push the first row of the result of columnToRow. 
    	\end{itemize}
    	\item Outside the function, return the newlist.
    	\item Test this function from main and the spreadsheet.  It should work exactly as the TRANSPOSE function did!
\end{itemize}

\section*{Submission}
When submitting the assignment, ensure that the settings are changed so that anyone with the link can view.
\begin{figure}[H]
  \centering
  \includegraphics{submission.png}
  \caption{Submission Settings}
\end{figure}

\section*{How this is graded}
This assignment is worth \AValue \ points. You will achieve all \AValue \   points if the following things are completed:
\begin{itemize}
    \item A rowToColumn function (25 pts)
    \item A selectColumn function (25 pts)
    \item A columnToRow function (25 pts)
    \item A transposeRange function (25 pts)
\end{itemize}
\end{document}