\documentclass{article}
\usepackage{fancyhdr}
\usepackage{float}
\usepackage[margin=1in]{geometry}
\usepackage{multicol}
\usepackage{url}
\usepackage{hyperref}
\usepackage{amsmath, amssymb, amsfonts}
\usepackage{graphicx}
\usepackage{xcolor}
\usepackage{subcaption}
\hypersetup{
    colorlinks=true,
    linkcolor=blue,
    urlcolor=blue,
}
\newcommand{\AName}{Scripting SWAT 12}
\newcommand{\ALength}{50 - 90 minutes}
\newcommand{\ADate}{12/05/2024}
\newcommand {\ADueDate}{12/10/2024}
\newcommand {\AAcceptDate}{12/12/2024}
\newcommand{\AValue}{100}
\pagestyle{fancy}
\fancyhead{}
\fancyhead[L]{\begin{tabular}{l}
	{\AName} \\
	{\ALength} \\
	\end {tabular}}
	
\fancyhead[C]{\begin{tabular}{|c|c|c|}
  \hline
  \textbf{Date Posted:} & \textbf{Date Due:} & \textbf{Accepted Until:} \\
  \hline
  \ADate & \ADueDate & \AAcceptDate \\
  \hline
  \end {tabular}}
  
\fancyhead[R]{\begin{tabular}{r}
	{COMP 110} \\
	{Fall 2024} \\
	\end {tabular}}

\begin{document}
\textbf{Welcome to \AName!  By the end of this lesson, Students Will be Able To...}
\begin{itemize}
    \item Be able to work with ranges in Apps Script
\end{itemize}


\section*{Part 1: Copy the file}
\begin{itemize}
    \item Open the following file:  \url{https://docs.google.com/spreadsheets/d/1jwJjscKkF-kRx2TEPHd2mMwb2u6BJtXzeZZ9jaWh5EQ/edit?usp=sharing}
    \item Once opened, click File, then Make a Copy.  Put this copy into your COMP 190 folder.
    \item Once copied, open the file and Apps Script.
\end{itemize}

\section*{Part 2: Sheets Review:  Ranges, the SUM function}
Let's first review the sum function and the concept of ranges from the spreadsheet.
\begin{itemize}
    \item In your Sheets file, in the range A1:D12, put in random numbers between 1 and 9.  Something like this:
    \begin{figure}[H]
  		\centering
  		\includegraphics{swat_12_sheet_range}
  		\caption{A range of numbers}
	\end{figure}
	\item This range of numbers has 4 columns and 6 rows.  You can reference each cell in the range using a combination of column first, then row.
	\item Think of each row in this range:  A1:D1, A2:D2, A3:D3, A4:D4, A5:D5, A6:D6.
	\item To the right of your range, in cell F1, use the SUM function to add up the numbers in the range.  It should look like this: =SUM(A1:D6).  Based on the example above, your sum should be 115.
	\item Let's think about what the SUM function is doing.  You are taking a range (many rows and many columns) and adding all the numbers within.
	\item We've been able to do this in Apps Script with lists, back in SWAT 6.  If you think of a list as a row, we know how to add 1 row, many columns.
	\item How do we now add many rows?  How does Apps Script "see" ranges?
\end{itemize}

\section*{Part 3: Ranges, Apps Script}
Let's look at ranges in Apps Script, and see how they are structured.
\begin{itemize}
    \item Using our main function, recreate each row of the spreadsheet example above.
    \begin{itemize}
    		\item A1:D1 can be written like this:  var r0 = [1, 2, 3, 4]
    		\item A2:D2 can be written like this:  var r1 = [4, 8, 9, 6]
    		\item A3:D3 can be written like this:  var r2 = [1, 4, 2, 3]
    		\item A4:D4 can be written like this:  var r3 = [7, 3, 7, 7]
    		\item A5:D5 can be written like this:  var r4 = [6, 6, 5, 1]
    		\item A6:D6 can be written like this:  var r5 = [7, 8, 9, 2]
    \end{itemize}
    \item Note that the spreadsheet starts position at one, but the script starts position at zero.  The sheet runs one through six, the script runs zero through five. 
    \item In main, create a variable called range, and make it a list.  Each item in our range list should be a row from above.  It should look like this:  var range = [r0, r1, r2, r3, r4, r5]
    \item In main, use console.log to show the range variable.  Save and run your main function.  You should get the following:
    \begin{figure}[H]
  		\centering
  		\includegraphics{swat_12_script_range}
  		\caption{A list of lists of numbers}
	\end{figure}
	\item This is the exact same thing as your ranges from part 2.  In Apps Script, a range is a list, of lists!  
\end{itemize}

\section*{Part 4: List of Lists, Properties and Navigation}
Let's look at some of the properties of a list of lists, compare them to the spreadsheet, and learn how to use loops to recreate references.
\begin{itemize}
    \item In your main function, use console.log to show the length of the range.  It should look like this:  console.log(range.length)
    \item Save and run your main function.  You should get a result of six.  Remember that the length is how many things are in the list.  In our list of lists, there are six things - each thing is a list.  This is just like rows in a spreadsheet.  range.length is the number of lists in a list, and is the same as the number of rows in a spreadsheet.
    \item If range.length is the number of rows, how do we find the number of columns?  That is easy - find the length of one row.  How to we address rows?  It is simple - use range[position] to show which row you wish to see.
    \item Change console.log to range[0].  Save and run your main function.  You should now get [1, 2, 3, 4]
    \item How do you find the length of that?  Simply add .length at the end.  It should look like this:  range[0].length 
    \item Save and run your main function.  You should get a result of 4.  This is your number of columns.  
    \item That's how to get the number of rows and columns.  How do we address each individual value in our range?  
    \item This is done with two coordinates after our list name.  For example, we used range[0] above to address the entire row.  To address a position within that row, we need a second [position] after our list name.
    \item For example, let's say we want to see that number 1 from our first row of [1, 2, 3, 4].  This is the 0 row, and we want the zero position within that row (aka the zero column).  Use range[0][0] to get that specific number.
    \item Modify your console.log to show range[0][0].  Save and run main - you should get 1.
    \item What is the result of range[0][1]?  If you said 2, you are correct! 
    \item What is the result of range[1][1]?  If you said 8, you are correct!
    \item In the range, there is a second 8.  What are the coordinates for it?  If you said range[5][1], you are correct!
    \item There is one five in the range.  What are the coordinates for it?  If you said range[4][2], you are correct!
    \item As you can see from the examples above, the coordinates are backwards from references in a sheet.  In a sheet, it's column first, then row.  In Apps Script, it's row first, then column. 
\end{itemize}

\section*{Part 5: displayRange function}
Let's set up a function that uses two loops to display a range.  
\begin{itemize}
    \item Create a new function called displayRange.  Send this function one argument - range.
    \item When called, the function should display the following (Note:  The entire output is not shown for space purposes):
    \begin{figure}[H]
  		\centering
  		\includegraphics{swat_12_display_range}
  		\caption{A list of lists of numbers}
	\end{figure}
	\item This prints each row, then each object within each row. 
    \item To accomplish this, we need two loops.  One needs to start at zero and run the number of rows (range.length).  Within, we will need a second loop that starts at zero and runs the number of columns (range[0].length).
    \item Within your displayRange function, create a loop variable, initially set to zero.
    	\item You will then need a loop that starts at zero, and traverses the number of rows in the range.  Use range.length to find the number of rows. Within the loop, do the following:
    	\begin{itemize}
    		\item Use console.log to display the entire row.  It should look something like this:  console.log(range[i])  
    		\item Increment your loop variable!
    	\end{itemize}
    	\item Save this function.  Return to your main function and call the displayRange function, sending the range created in part 3.  You should get each row individually printed out.
    	\item Return to your displayRange function.  You will now need a second loop.  
    	\item After the console.log(range[i]) line, create a second loop variable, initially set to zero.  
    	\item You will need a loop that starts at zero and runs for the number of columns.  Use range[0].length for the number of columns.  Within this loop, do the following:
    	\begin{itemize}
    		\item Use console.log to display each object in the range.  You will need to address each object by row first, then column.  It should look something like this:  console.log(range[i][j])  
    		\item Increment your loop variable!
    	\end{itemize}
    	\item Save this function, and run your main function once again.  You should now see each object in the range, listed individually.
    	\item You now know the basic setup for working with ranges.  Let's use this to add up the numbers in a range! 
\end{itemize}

\section*{Part 6: sumRange function}
Let's write our own version of the SUM function.
\begin{itemize}
    \item Copy the displayRange function from part 5.  Paste it, and rename the function to sumRange.  You will need the same argument - range.
    \item As this is adding up numbers, we will need a sum variable, initially set to zero.
    \item As we have the basic structure, we now need to make a few small modifications in order to sum the numbers.  Make the following changes:
    \begin{itemize}
    		\item Remove the console.log(range[i]) line.
    		\item Replace the console.log(range[i][j] line with your summation instead.  You will need to add the number at each position in the range, starting with row first, then column.  It should look like this:  sum = sum + range[i][j]
    		\item Outside of the loops, return the sum.
    	\end{itemize}
    	\item Save this, and call this function from your main function.  Use console.log(sumRange(range)) to do so.  If you're using the example from above, you should get 115.
    	\item Once it works in Apps Script, test this function from the sheet.  You should get the same answer as the SUM function!
\end{itemize}

\section*{Part 7: sumRange2 function}
We know how to add numbers in a range, using two loops.  How can we do the same thing, only using two functions with one loop each?
\begin{itemize}
    \item Create a sumRange2 function.  Send this function one argument - range.
    \item Harken back to SWAT 6, when we first started working on lists.  We wrote a function called sumList, that adds up all the numbers in a list.  This function is included in the start file.
    \item In your sunRange2 function, do the following:
    \begin{itemize}
    		\item Set up a sum variable, initially set to zero.
    		\item Set up a loop variable, initially set to zero.
    		\item You'll need a loop that runs the number of rows in the range.  Within the loop, do the following:
    		\begin{itemize}
    			\item Sum the numbers in each row of the range by calling the sumList function.  You'll need to send each individual row of the range as the argument for sumList.
    			\item Increment your loop variable!
    		\end{itemize}
    		\item Outside your loop, return the sum.
    	\end{itemize}
    	\item Save and test this function from main.  Once it works in Apps Script, test it using the sheet.
\end{itemize}

\section*{Part 8: rangeToRow function}
There are times when you will need to turn a range into a list.  Let's write a function to do so!
\begin{itemize}
	\item  Copy your displayRange function, and rename it to rangeToRow.  You will send this function one argument - range.
	\item Modify your rangeToRow function as follows:
	\begin{itemize}
		\item Remove the console.log lines.
		\item You will need a new blank list.
		\item Loop through the range, pushing each object into the newlist.
		\item Return your new list at the end of the function.  When doing so, return the newlist as a list of lists, like this:  return [newlist]
	\end{itemize}
	\item This function should work both in Apps Script and when called from the sheet.
\end{itemize}

\section*{Submission}
When submitting the assignment, ensure that the settings are changed so that anyone with the link can view.
\begin{figure}[H]
  \centering
  \includegraphics{submission.png}
  \caption{Submission Settings}
\end{figure}

\section*{How this is graded}
This assignment is worth \AValue \ points. You will achieve all \AValue \   points if the following things are completed:
\begin{itemize}
    \item A displayRange function (25 pts)
    \item A sumRange function (25 pts)
    \item A sumRange2 function (25 pts)
    \item A rangeToRow function (25 pts)
\end{itemize}
\end{document}