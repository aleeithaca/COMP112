\documentclass{article}
\usepackage{fancyhdr}
\usepackage{float}
\usepackage[margin=1in]{geometry}
\usepackage{multicol}
\usepackage{url}
\usepackage{hyperref}
\usepackage{amsmath, amssymb, amsfonts}
\usepackage{graphicx}
\usepackage{xcolor}
\usepackage{subcaption}
\hypersetup{
    colorlinks=true,
    linkcolor=blue,
    urlcolor=blue,
}
\newcommand{\AName}{Scripting SWAT 7}
\newcommand{\ALength}{50 - 90 minutes}
\newcommand{\ADate}{11/12/2024}
\newcommand {\ADueDate}{11/14/2024}
\newcommand {\AAcceptDate}{11/19/2024}
\newcommand{\AValue}{100}
\pagestyle{fancy}
\fancyhead{}
\fancyhead[L]{\begin{tabular}{l}
	{\AName} \\
	{\ALength} \\
	\end {tabular}}
	
\fancyhead[C]{\begin{tabular}{|c|c|c|}
  \hline
  \textbf{Date Posted:} & \textbf{Date Due:} & \textbf{Accepted Until:} \\
  \hline
  \ADate & \ADueDate & \AAcceptDate \\
  \hline
  \end {tabular}}
  
\fancyhead[R]{\begin{tabular}{r}
	{COMP 110} \\
	{Fall 2024} \\
	\end {tabular}}

\begin{document}
\textbf{Welcome to \AName!  By the end of this lesson, Students Will be Able To...}
\begin{itemize}
    \item Utilize lists to analyze words, aka strings of characters
	\item Write unique and interesting functions that don't exist in the spreadsheet.
\end{itemize}


\section*{Part 1: Create the file}
\begin{itemize}
    \item In your COMP-190 folder, create a new Google Sheets file.  Rename this file to SWAT 7.  Open Apps Script.  Rename the project to SWAT 7.
\end{itemize}

\section*{Part 2: Words as Lists}
Computers don't see words; computers see "strings" of characters.  These "strings" are also seen as lists of individual characters.
\begin{itemize}
    \item Create a new function called main, no arguments.  
    \item Within main, create a new variable called word.  Assign this variable the text "banana"
    \item Use console.log to show word[0].  What character is shown?
    \item What would you do to show the first "n"?
    \item Two ways of showing word:  var word = "banana" or var word = ["b","a","n","a","n","a"]
\end{itemize}

\section*{Part 3: Vowels}
How many vowels are in banana?
\begin{itemize}
    \item Let's review vowels before we start writing code.  Vowels are a, e, i, o and u.
    \item y is sometimes a vowel.  If there are no other vowels present, y is a vowel.  If any of the other vowels are present, y is not a vowel.
    \item For example, the y in Monday is not a vowel - there's an o and a there.  In the word myth, y is a vowel, as there are no other vowels.
    \item How do we count vowels in Sheets?  A quick search reveals this:  \url{https://www.excelforum.com/excel-general/592300-counting-vowels.html}
    \item Take a look at the formulas in that page.  Complex, not intuitive, and the sheet syntax makes things difficult.  It also does not account for the "sometimes y" portion of vowels!
    \item Keep this in mind when writing our next function.  With a few lines of code, we can easily write a function to count vowels, and account for y!
\end{itemize}

\section*{Part 4: vowelCount function}
Let's write a function that counts the vowels in a word!
\begin{itemize}
    \item Create a new function called vowelCount.  Send this function one argument - word.
    \item Our function will need the following:
    \begin{itemize}
    		\item A loop variable that starts at zero.
    		\item A count variable that starts at zero.
    		\item A loop that starts at zero and runs the length of the word
    	\end{itemize}
    	\item Within your loop, to the following:
    	\begin{itemize}
    		\item See if the current list position is the same as "a".  If it is, increment count.
    		\item See if the current list position is the same as "e".  If it is, increment count.
    		\item See if the current list position is the same as "i".  If it is, increment count.
    		\item See if the current list position is the same as "o".  If it is, increment count.
    		\item See if the current list position is the same as "u".  If it is, increment count.
    	\end{itemize}
    	\item Once the loop is finished, do the following:
    	\begin{itemize}
    		\item See if the count is zero.  If it is, run the loop again, this time, counting the y character.
    		\item Return the count.
    	\end{itemize}
    	\item Use a main function and the spreadsheet to find out of this works.
\end{itemize}

\section*{Part 5: vowelCount2 function}
The function above takes 2 loops.  How do we do it in one?
\begin{itemize}
    \item Copy the vowelCount function.  Paste it.  Rename the function to vowelCount2.
    \item What changes need to be made to run a loop only once?
    \item Here's a hint:  2 counting variables.
    \item Use the main function and spreadsheet to make sure this works.
\end{itemize}

\section*{Submission}
When submitting the assignment, ensure that the settings are changed so that anyone with the link can view.
\begin{figure}[H]
  \centering
  \includegraphics{submission.png}
  \caption{Submission Settings}
\end{figure}

\section*{How this is graded}
This assignment is worth \AValue \ points. You will achieve all \AValue \   points if the following things are completed:
\begin{itemize}
    \item A vowelCount function (50 pts)
    \item A vowelCount2 function (50 pts)
\end{itemize}
\end{document}