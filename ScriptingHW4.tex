\documentclass{article}
\usepackage{fancyhdr}
\usepackage{float}
\usepackage[margin=1in]{geometry}
\usepackage{multicol}
\usepackage{url}
\usepackage{hyperref}
\usepackage{amsmath, amssymb, amsfonts}
\usepackage{graphicx}
\usepackage{xcolor}
\usepackage{subcaption}
\hypersetup{
    colorlinks=true,
    linkcolor=blue,
    urlcolor=blue,
}
\newcommand{\AName}{Scripting HW 4}
\newcommand{\ALength}{50 - 90 minutes}
\newcommand{\ADate}{03/31/2025}
\newcommand {\ADueDate}{04/02/2025}
\newcommand {\AAcceptDate}{04/07/2025}
\newcommand{\AValue}{100}
\pagestyle{fancy}
\fancyhead{}
\fancyhead[L]{\begin{tabular}{l}
	{\AName} \\
	{\ALength} \\
	\end {tabular}}
	
\fancyhead[C]{\begin{tabular}{|c|c|c|}
  \hline
  \textbf{Date Posted:} & \textbf{Date Due:} & \textbf{Accepted Until:} \\
  \hline
  \ADate & \ADueDate & \AAcceptDate \\
  \hline
  \end {tabular}}
  
\fancyhead[R]{\begin{tabular}{r}
	{COMP 110} \\
	{Spring 2025} \\
	\end {tabular}}

\begin{document}
\textbf{Welcome to \AName!  By the end of this lesson, Students Will be Able To...}
\begin{itemize}
    \item Utilize ideas from SWAT 4 to create functions to count even and odd numbers.
\end{itemize}


\section*{Part 1: Create the file}
\begin{itemize}
    \item In your COMP-190 folder, create a new Google Sheets file.  Rename this file to HW 4.  Open Apps Script.  Rename the project to HW 4.
    \item Copy the isEven function from your SWAT 4 assignment into your HW 4 project.
\end{itemize}

\section*{Part 2: countNums function}
Here we will reuse ideas from SWAT 4 to create a new function.  This function will count the number of even or odd numbers, from a starting number to an ending number.
\begin{itemize}
	\item Create a new function called countNums.  This function should count either the number of even or all of the odd numbers, between 2 numbers.  This function should take three arguments - start, end, and type.  The arguments are as follows:
	\begin{itemize}
		\item start - a starting number.
		\item end - an ending number.
		\item type - this is odd or even.  The result of the function is either the number of even number, or the number of odd numbers, depending on which is specified.
	\end{itemize}
	\item The function should do the following:
	\begin{itemize}
		\item If the type is even, the function should return the number of even numbers between start and end.
		\item If the type is odd, the function should return the number of odd numbers between start and end.
		\item If the end number is less than the start number, the function should return an error message stating that the end number should be greater than the start number.
		\item If the type is not even or odd, return an error message stating to use a type of even or odd.
		\item Use the isEven function from SWAT 4 to check and see if a number is even or odd.
		\item Use a main function to test the countNums function.
		\item Use the spreadsheet to call the countNums function.
	\end{itemize}
	\item See the screenshot below for an example.
	\begin{figure}[H]
  		\centering
  		\includegraphics{hw_4_countnums}
  		\caption{Results from the countNums function}
	\end{figure}
\end{itemize}

\section*{Part 3: findFact function}
Similar to the FACT function in Google Sheets, we can now use a loop to find the factorial of a number.  We will write our own function called findFact to do so.
\begin{itemize}
	\item The factorial of a number $n$ is the numbers 1 through n, multiplied together.  Examples are below.
	\begin{itemize}
		\item The factorial of 2 is 1 * 2.
		\item The factorial of 3 is 1 * 2 * 3.
		\item The factorial of 4 is 1 * 2 * 3 * 4.
		\item The factorial of 5 is 1 * 2 * 3 * 4 * 5.  
		\item The factorial of zero and 1 is 1.  
		\item You cannot find the factorial of negative numbers.
	\end{itemize}
	\item Create a new function called findFact.  Send this function a number, n.
	\item Within your function, you will need the following:
	\begin{itemize}
		\item if n $<$ 0, display an error message stating that n must be greater than zero.
		\item You will need a variable called total, to store the total of the numbers multiplied.
		\item Use a loop to multiply your numbers.
		\item Use the main function to test your function in Apps Script.
		\item Call your function from the spreadsheet.
	\end{itemize}
	\item See the screenshot below.
	\begin{figure}[H]
  		\centering
  		\includegraphics{hw_4_factorial}
  		\caption{Results of the findFact function}
	\end{figure}
\end{itemize}

\section*{Submission}
When submitting the assignment, ensure that the settings are changed so that anyone with the link can view.
\begin{figure}[H]
  \centering
  \includegraphics{submission.png}
  \caption{Submission Settings}
\end{figure}

\section*{How this is graded}
This assignment is worth \AValue \ points. You will achieve all \AValue \   points if the following things are completed:
\begin{itemize}
    \item A countNums function that counts even or odd numbers, complete with error messages (50 pts)
    \item A findFact function that finds the factorial of a number. (50 pts)
\end{itemize}
\end{document}