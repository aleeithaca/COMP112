\documentclass{article}
\usepackage{fancyhdr}
\usepackage{float}
\usepackage[margin=1in]{geometry}
\usepackage{multicol}
\usepackage{url}
\usepackage{hyperref}
\usepackage{amsmath, amssymb, amsfonts}
\usepackage{graphicx}
\usepackage{xcolor}
\usepackage{subcaption}
\hypersetup{
    colorlinks=true,
    linkcolor=blue,
    urlcolor=blue,
}
\newcommand{\AName}{Scripting HW 7}
\newcommand{\ALength}{50 - 90 minutes}
\newcommand{\ADate}{11/12/2024}
\newcommand {\ADueDate}{11/14/2024}
\newcommand {\AAcceptDate}{11/19/2024}
\newcommand{\AValue}{100}
\pagestyle{fancy}
\fancyhead{}
\fancyhead[L]{\begin{tabular}{l}
	{\AName} \\
	{\ALength} \\
	\end {tabular}}
	
\fancyhead[C]{\begin{tabular}{|c|c|c|}
  \hline
  \textbf{Date Posted:} & \textbf{Date Due:} & \textbf{Accepted Until:} \\
  \hline
  \ADate & \ADueDate & \AAcceptDate \\
  \hline
  \end {tabular}}
  
\fancyhead[R]{\begin{tabular}{r}
	{COMP 110} \\
	{Fall 2024} \\
	\end {tabular}}

\begin{document}
\textbf{Welcome to \AName!  By the end of this lesson, Students Will be Able To...}
\begin{itemize}
    \item Write functions similar to COUNTIF, SUMIF and AVERAGEIF
\end{itemize}


\section*{Part 1: Create the file}
\begin{itemize}
    \item In your COMP-190 folder, create a new Google Sheets file.  Rename this to HW 7.  Open Apps Script.  Rename the project to HW 7.
\end{itemize}

\section*{Part 2: COUNTIF, SUMIF, AVERAGEIF review}
Let us begin by looking at the spreadsheet functions of COUNTIF, SUMIF and AVERAGEIF and observing how they work.
\begin{itemize}
    \item Go to your HW 7 spreadsheet.  In the range A1:J1, type the numbers 1 through 10.
    \item Recall the COUNTIF function.  The syntax of this function is the following:  COUNTIF(range, criterion).
    \item The range is where to look.  The criterion is what to look for.  In this case, our range is A1:J1, and we are looking for numbers greater than 5.
    \item SUMIF and AVERAGEIF have the same arguments - where to look and what to look for.  Since where we are looking and what we are adding/averaging are the same, no additional arguments are necessary.
    \item Below your numbers, set up the sheet with the following:
    \begin{itemize}
    		\item 3 cells labeled COUNTIF, SUMF and AVERAGEIF
    		\item Next to those cells, the COUNTIF, SUMIF and AVERAGEIF function, all looking for numbers greater than 5.
    	\end{itemize}
    	\item Your sheet should look like this:
    	\begin{figure}[H]
  		\centering
  		\includegraphics{hw_6_ss_review}
  		\caption{COUNTIF, SUMIF, AVERAGEIF Review}
	\end{figure}
	\item We will have to do things slightly differently when writing our own functions.
	\item Instead of a range, we will send the function a list. 
	\item Note the criterion argument - it's got the operator ($<$, $>$, $=$, $>=$, $<=$, $!=$) and the value put together.
	\item For example, "$>=$5" tells the function to count the numbers greater than or equal to 5.
	\item Our function will need to send the criterion as a separate argument.  
	\item For example, our function will look something like this:  function countIfList(list, "$>=$", 5)
	\item The other constraint on our functions is that our lists will be numbers only - no text!
\end{itemize}

\section*{Part 3: main function}
To test our functions, let's set up the main function with our variables.
\begin{itemize}
    \item Create a new function called main, with no arguments.
    \item Within main, create the following three variables:
    \begin{itemize}
    		\item var list = [1, 2, 3, 4, 5, 6, 7, 8, 9, 10]
    		\item var method = "$<$"
    		\item var value = 5
    	\end{itemize}
    	\item Use console.log to show the list.  Run the main function to make sure it works.
\end{itemize}

\section*{Part 4: countIfList function}
We will mimic the COUNTIF function by creating our own function, countIfList.
\begin{itemize}
    \item Create a new function called countIfList.  This function needs three arguments - list, method and value (you can use l, m and v for brevity's sake, if you so choose).
    \item You will need the following variables:
    \begin{itemize}
    		\item a Loop variable (most likely i), initially set to zero.
    		\item A variable called count, initially set to zero.
    	\end{itemize}
    	\item You will need a loop.  The loop should start at zero and go the entire length of the list.
    	\item Within the loop, do the following:
    	\begin{itemize}
    		\item Check to see if method (or m) is equal to the text "="
    		\item If this is true, check to see if value is equal to the current list position.
    		\item If this is true, increment count by 1.
    		\item DO NOT USE ELSE - check to see if method is equal to the text "!="
    		\item If this is true, check to see if value is equal to the current list position.
    		\item If this is true, increment count by 1.
    		\item DO NOT USE ELSE - check to see if method is equal to the text "$<$"
    		\item If this is true, check to see if current list position is less than value.
    		\item If this is true, increment count by 1.
    		\item DO NOT USE ELSE - check to see if method is equal to the text "$<=$"
    		\item If this is true, check to see if current list position is less than or equal to value.
    		\item If this is true, increment count by 1.
    		\item DO NOT USE ELSE - check to see if method is equal to the text "$>$"
    		\item If this is true, check to see if current list position is greater than value.
    		\item If this is true, increment count by 1.
    		\item DO NOT USE ELSE - check to see if method is equal to the text "$>=$"
    		\item If this is true, check to see if current list position is greater than or equal to value.
    		\item If this is true, increment count by 1.
    		\item DO NOT WORRY ABOUT ERROR MESSAGES OR ELSE.
    		\item Don't forget to increment your loop variable by 1!
    	\end{itemize}
    	\item Outside of the loop, return count.
    	\item Test this to make sure it works!  Call the function from main.  It will not work from the sheet at the moment.
\end{itemize}

\section*{Part 4: sumIfList function}
Similar to the countIfList function, we will utilize the same ideas to sum the numbers in a list.
\begin{itemize}
    \item Copy your countIfList function.  Rename this function to sumIfList.
    \item In your function, change your count variable to sum.
    \item Instead of incrementing count by 1, add the current list position to sum.
    \item Return sum instead of count.
    \item Use the main function to test it.
\end{itemize}

\section*{Part 5:avgIfList function}
This function will utilize the countIfList and sumIfList functions to return an average.
\begin{itemize}
    \item Create a new function called avgIfList.  Send this function three arguments - list, method, and value.
    \item This function needs to do one thing:  return sumifList divided by countIfList.
    \item Use the main function to test it.
\end{itemize}

\section*{Submission}
When submitting the assignment, ensure that the settings are changed so that anyone with the link can view.
\begin{figure}[H]
  \centering
  \includegraphics{submission.png}
  \caption{Submission Settings}
\end{figure}

\section*{How this is graded}
This assignment is worth \AValue \ points. You will achieve all \AValue \   points if the following things are completed:
\begin{itemize}
    \item A countIfList function (45 pts)
    \item a sumIfList function (45 pts)
    \item An avgIfList function (10 pts)
\end{itemize}
\end{document}